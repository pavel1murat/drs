%%%%%%%%%%%%%%%%%%%%%%%%%%%%%%%%%%%%%%%%%%%%%%%%%%%%%%%%%%%%%%%%%%%%%%%%%%%%%%%
% files to use
% 

%\documentclass[aps,prd,showpacs,superscriptaddress,twocolumn]{revtex4}  % for submission
\RequirePackage{lineno}
\documentclass[aps,prd,showpacs,preprint,groupedaddress]{revtex4}  % for review
%\documentclass[12pt]{article} %{revtex4}  % for review
%
%%%%%%%%%% \documentclass[aps,preprint,showpacs,superscriptaddress]{revtex4}  % for submission
\usepackage{graphicx}  % needed for figures
\usepackage{epsf}  % needed for figures
\usepackage{dcolumn}   % needed for some tables

\usepackage{setspace}  % for doublespacing
%\linespread{2.}

\usepackage{bm}        % for math
\usepackage{ulem}      % for strike out
\usepackage{color}     %
\usepackage{amssymb}   % for math
%\usepackage{multicol}   % needed for some tables
\usepackage{multirow}
\usepackage{epstopdf}  %added for MAC compiler

%%%%%%%%%%%%%%%%%%%%%%%%%%%%%%%%%%%%%%%%%%%%%%%%%%%%%%%%%%%%%%%%%%%%%%%%%%%%%%%
\begin{document}
%%%%%%%%%%%%%%%%%%% \input{psfig}
% \input{commands} 
%%%%%%%%%%%%%%%%%%%%%%%%%%%%%%%%%%%%%%%%%%%%%%%% local new commands

\newcommand {\chisq}         {\mbox{$\chi^2$}}
\newcommand {\rsw}           {\mbox{$R_{sw}$}}
\newcommand {\rwp}           {\mbox{$R_{wp}$}}
\newcommand {\DeltaPhi}      {\mbox{$\Delta\phi$}}
\newcommand {\ecal}          {\mbox{$E_{cal}$}}
\newcommand {\et}            {\mbox{${E_T}$}}
\newcommand {\etcal}         {\mbox{${E_T}_{cal}$}}
\newcommand {\etvis}         {\mbox{${E_T}_{vis}$}}
\newcommand {\evis}          {\mbox{$E_{vis}$}}
\newcommand {\ewire}         {\mbox{$E_{wire}$}}
\newcommand {\etauid}        {\mbox{$\epsilon_{ID}^{\tau}$}}
\newcommand {\estrip}        {\mbox{$E_{strip}$}}
\newcommand {\etcorr}        {\mbox{$E_T^{corr}$}}
\newcommand {\gt}            {\mbox{$>$}}
\newcommand {\gea}           {\mbox{$>=$}}
\newcommand {\invfb}         {\mbox{$fb^{-1}$}}
\newcommand {\invpb}         {\mbox{$pb^{-1}$}}
\newcommand {\LaBrThree}     {\mbox{$\rm LaBr_3$}}
\newcommand {\lea}           {\mbox{$<=$}}
\newcommand {\lt}            {\mbox{$<$}}
\newcommand {\met}           {\mbox{${\not \! E}_{T}$}}

\newcommand {\mmsq}          {\mbox{${\rm mm^2}$}}
\newcommand {\mmcube}        {\mbox{${\rm mm^3}$}}

\newcommand {\NtrkTen}       {\mbox{$N_{trk}^{10}$}}
\newcommand {\NtrkTenThirty} {\mbox{$N_{trk}^{10-30}$}}
\newcommand {\ppbar}         {\mbox{$p\bar{p}$}}
\newcommand {\ttbar}         {\mbox{$t\bar{t}$}}
\newcommand {\pt}            {\mbox{$p_{T}$}}
\newcommand {\ptrack}        {\mbox{$P_{track}$}}
\newcommand {\vismass}       {\mbox{$M_{track+\pi^0}$}}
\newcommand {\tzeroa}        {\mbox{$T_0^a$}}
\newcommand {\yrec}          {\mbox{$Y_{rec}$}}
\newcommand {\zrec}          {\mbox{$Z_{rec}$}}

\newcommand{\Pt}{\ensuremath {p_{\rm{t}}}}
\newcommand{\Et}{\ensuremath{E_{T}}}

%%% \newcommand {\upar}               {\mbox{$U_{||}$}}
%%%% $ \sum \vec{p_t} $

\newcommand {\plots} {/home/murat/figures/drs4}

%%%%%%%%%%%%%%%%%%%%%%%%%%%%%%%%%%%%%%%%%%%%%%%%%%%%%%%%%%%%%%%%%%%%%%%%%%%%%%%
\begin{titlepage}

  \begin{flushright}
    \bf Fermilab Technical note  \\
    draft version 1.02
    \today
  \end{flushright}
  
  \vspace{1cm}
  
  \begin{center} {
      \Large \bf
      Study of the timing resolution of the LYSO crystals with the SiPM-based readout
    }

    \vspace{1cm}

    { 
      P.Murat(Fermilab)
    }

 \end{center}

  \vspace{1cm}

  \begin{abstract}
    Results of the study, look at the method

  \end{abstract}

\end{titlepage} 

{\tableofcontents}

%%%%%%%%%%%%%%%%%%%%%%%%%%%%%%%%%%%%%%%%%%%%%%% Introduction
\section{Introduction}

In this note we document reconstruction and analysis procedures used to analyse
data from the PET-TOF test stand.

%%%%%%%%%%%%%%%%%%%%%%%%%%%%%%%%%%%%%%%%%%%%%%%%%%%%%%%%%%%%%%%%%%%%%%%%%%%%%%%
\section{Experimental Setup}

The experimental setup consists of a box-shaped scintillating crystal 
read out from the two opposite sides by the SiPM matrices. 

The coordinate system is chosen such that the SiPM matrices are located on the 
two Z-sides of the crystal, as shown in Fig.~\ref{fig:crystals}.

\begin{figure}[h!]
  \begin{center}
%    \vspace*{-2.8cm}
    \includegraphics [width=.9\textwidth, clip=true, viewport=0.in .0in 8.in 8.in] 
       {\plots/missing_plot}
       \caption[]{Experimental setup: crystals and SiPMs}
       \label{fig:crystals}
  \end{center}
\end{figure}

%%%%%%%%%%%%%%%%%%%%%%%%%%%%%%%%%%%%%%%%%%%%%%%%%%%%%%%%%%%%%%%%%%%%%%%%%%%%%%%
\section{Energy resolution}

The waveform processing starts from finding the pedestal, or the offset voltage, 
and subtracting it from the waveform. The pedestal subtraction procedure is described 
in Section \ref{sec:appendix_1}

The pulse charge, a measure of the incoming photon energy, is calculated by 
integrating the pulse over 40ns (200 channels) and normalizing the photopeak 
in the distribution in Fig.~\ref{fig:run_072_waveform_ph_q1} to the expected 
energy of the non-scattered annihilation photon, 511 KeV. The energy resolution 
in the peak is $\Delta E / E ~\simeq~ 9\%$ and $\simeq~ 9.4\%$ FWHM for channels 0 
and 1 correspondingly. This resolution corresponds to about 650 photons registered 
by the SiPM, or, overall photodetection efficiency which includes effects of 
light reflection, collection and SiPM quantum efficiency of about 5\%.

Alternatively, the photon energy can also be measured using the pulse height 
in which case the corresponding resolutions are $\sim 22\%$ and $\sim 23\%$ 
for channels 0 and 1, or worse by about a factor of 2.
\begin{figure}[h!]
  \begin{center}
%    \vspace{-3.0cm}
    \includegraphics [width=1.\textwidth, clip=true, viewport=0.in .0in 8.in 8.in] 
       {\plots/srcs01_000072/run_072_waveform_ph_q1}
       \caption[]{
         Top: the signal waveforms, the typical pulse length of about 40 ns; 
         Middle: distribution in the pulse height ad integrated charge, normalized to 511KeV 
         for channel \#0; 
         Bottom: distributions in the pulse height and the normalized integrated charge 
         for channel \#1
       }
       \label{fig:run_072_waveform_ph_q1}
  \end{center}
\end{figure}


%%%%%%%%%%%%%%%%%%%%%%%%%%%%%%%%%%%%%%%%%%%%%%%%%%%%%%%%%%%%%%%%%%%%%%%%%%%%%%
\newpage
\section{Fitting }

There is no analythical function which could be used as a model of a pulse 
leading egde based on the apriori considerations, so in most cases the 
parameterization choice is driven by the data. 

%%%%%%%%%%%%%%%%%%%%%%%%%%%%%%%%%%%%%%%%%%%%%%%%%%%%%%%%%%%%%%%%%%%%%%%%%%%%%%%
\section{Gaussian Fit}
\label{sec:gaussian_fit}

The idea of parameterizing the leading edge of a pulse with a gaussian comes from 
the Fig. \ref{fig:run_72_ev_105_gaus_shape_display} which shows that the 
significant part of the waveform leading edge can be well described by the gaussian.

\begin{figure}[h!]
  \begin{center}
%    \vspace*{-2.8cm}
    \includegraphics [width=.9\textwidth, clip=true, viewport=0.in .0in 8.in 6.in] 
       {\plots/srcs01_000072/run_72_ev_105_gaus_shape_display}
       \caption[]{
         Gaussian fits in the range of 0.05-0.5 of the pulse height \\
         {\bf 1. describe the error bars}
       }
       \label{fig:run_72_ev_105_gaus_shape_display}
  \end{center}
\end{figure}


The most common hardware-based time measurement technique, based on the use 
of Constant Fraction Discriminators (CFD), determines the pulse time stamp $T_0$ 
as the time, when the pulse, developing in time, reaches certain fraction of 
its maximal height, typically 10\%. This technique is based on the assumption that 
the pulse leading edge has constant shape, such that $\Delta T ~=~ T_0^{CDF}-T_0$  
is a constant. However, statistical fluctuations at the early part of pulse development 
could affect the timing resolution. 

Gaussian parameterization of the waveform, $V(t) ~=~ A e^{-(t-t_0)^2/\sigma_t^2}$, 
doesn't have a parameter, which is directly related with $T_0^{pulse}$ or could 
be translated into it. For that reason, the pulse timestamp $T_0$ is calculated 
using the following ansatz:
$$
      T_0 ~=~ T(\alpha V_{max})  -  {{ \alpha V_{max}} \over { dV/dT(\alpha V_{max})}}
$$

as shown in Fig.\ref{fig:gaussian_pulse_t0}. A data-driven optimization of the timing 
resolution resulted in $\alpha = 0.07$ 

{\bf 1. do we need to show the optimization curve ? }

\begin{figure}[h!]
  \begin{center}
%    \vspace*{-2.8cm}
    \includegraphics [width=.9\textwidth, clip=true, viewport=0.in .0in 8.in 8.in] 
       {\plots/missing_plot}
       \caption[]{T0 determination for gaussian parameterization}
       \label{fig:gaussian_pulse_t0}
  \end{center}
\end{figure}

For the fit, an uncertainty in the measured voltage could be considered as coming 
from the two sources: random noise and the shower development statistics. 

$$
     \sigma_i ~=~ 1.5 \sigma_{noise} + + 0.05 V_i;
$$

{\bf Obviously, the statistical term should be proportional to $\sqrt{V_i}$ - check that ...}

The distributions in the fit chi2 are presented in Fig. \ref{fig:srcs01_000072_fig_1}

\begin{figure}[h!]
  \begin{center}
%    \vspace*{-2.8cm}
    \includegraphics [width=.9\textwidth] % , clip=true, viewport=0.in .0in 8.in 5.in] 
       {\plots/srcs01_000072/note/fig_1}
       \caption[]{
         Gaussian fits: distributions in $\chi^2/N_{DOF}$ and the number of fit points 
       }
       \label{fig:srcs01_000072_fig_1}
  \end{center}
\end{figure}

To see what are advantages of the pulse fitting, in Fig. \ref{fig:resolution_vs_chi2_gaus} 
we compare the resolution in $T_0^{gaus}$ to the resolution in $T_0^{CFD}$, where CFD stands 
for a Constrant Fraction Discriminator.

Plotted in Fig. \ref{fig:srcs01_000072_fig_2} is the distribution in the gaussian sigma, returned 
by the fit. The mean of the distribution is about 0.8 ns, which translates into 
the pulse rise time, from 10\% to 90\%, of about 1.6 ns.

\begin{figure}[h!]
  \begin{center}
%    \vspace*{-2.8cm}
    \includegraphics [width=1.0\textwidth] %, clip=true, viewport=0.in .0in 8.in 8.in] 
       {\plots/srcs01_000072/note/fig_2}
       \caption[]{Gaussian Sigma - a fit parameter, the distribution mean is of the order of 800 ps }
       \label{fig:srcs01_000072_fig_2}
  \end{center}
\end{figure}

Fig.\ref{fig:srcs01_000072_fig_4} shows distribution in ``reduced T0'', fractional part 
of T0. A bias with respect to the first fit point may reveal itself as a non-uniformity in 
this distribution. We do not observe statistically significant nonuniformities.

\begin{figure}[h!]
  \begin{center}
%    \vspace*{-2.8cm}
    \includegraphics [width=1.0\textwidth] %, clip=true, viewport= 0.in .0in 8.in 7.in] 
       {\plots/srcs01_000072/note/fig_4}
       \caption[]{
         Distribution in fractional part of fitted T0. As times corresponding 
         to fit points are forced to the corresponding bin centers, a bias caused 
         by this choice may result in non-uniform distribution in the found T0 
         with respect to the bin center. The distributions are consistent with 
         being uniform.
       }
       \label{fig:srcs01_000072_fig_4}
  \end{center}
\end{figure}

{\bf TGausFitAlg} performs fit with equal weights (``w0q'' options of TH1::Fit).
The \chisq\ per degree of freedom, obviously, comes out very small. 
I'm trying to renormalize it by doing something very questionable:
$$
         \chisq ~=~ \chisq \cdot V_{max}^2
$$

the resulting selection in {\bf drsana} histogrammming is based on this 
definition of the \chisq. Effectively, the \chisq\ cut selects smaller pulses.

I'm not using this \chisq\ definition in any other place. For the same charge,
smaller pulse height means wider pulse.

{\bf Is there any correlation between the pulse height and the gaussian width? }

Timing resolution vs chi2 cut for gaussian fit is shown in Fig. \ref{fig:srcs01_000072_fig_3}.
One can see that, compared to the ``constant fraction'' technique, 
using the waveform parameterization and fitting the pulse leading edge allows to improve 
the coincidence timing resolution by about 15\%.

\begin{figure}[h!]
  \begin{center}
%    \vspace*{-2.8cm}
    \includegraphics [width=1.0\textwidth] %, clip=true, viewport=0.in .0in 8.in 8.in] 
       {\plots/srcs01_000072/note/fig_3}
       \caption[]{Timing coincidence resolution : GAUS vs CFD}
       \label{fig:srcs01_000072_fig_3}
  \end{center}
\end{figure}


%%%%%%%%%%%%%%%%%%%%%%%%%%%%%%%%%%%%%%%%%%%%%%%%%%%%%%%%%%%%%%%%%%%%%%%%%%%%%%%
\newpage
\section{Exponential Fit}

Depending on how well the parameterization reflects the physics underlying 
the pulse development in time, different parameterizations may result in 
different timing resolution. 

The exponential parameterization is inspired by the consideration that the 
early development of the avalanche should be exponential in time, 
supported by the Fig. \ref{fig:run_72_ev_105_exp1_shape_display}, 
showing exponential fit of the pulses for one of events. 

\begin{figure}[h!]
  \begin{center}
%    \vspace*{-2.8cm}
    \includegraphics [width=1.0\textwidth] %, clip=true, viewport=0.in .0in 8.in 8.in] 
       {\plots/srcs01_000072/run_72_ev_105_exp1_shape_display}
       \caption[]{fig 10}
       \label{fig:run_72_ev_105_exp1_shape_display}
  \end{center}
\end{figure}

The early, 0.02-0.25, part of the waveform is fit with the exponential 
$$
         A(t) = \alpha e^{ \beta(t-T_0) } ~{\rm for~} ~t > T_0 ~;~ A(t) = 0 ~~{\rm for~} ~ t < T_0
$$

and the fit parameter $T_0$ can be interpreted as the pulse start time.


Distribuions  in the fitted parameters are shown in Fig. \ref{fig:srcs01_000072_fig_45}.

\begin{figure}[h!]
  \begin{center}
%    \vspace*{-2.8cm}
    \includegraphics [width=1.0\textwidth] % , clip=true, viewport=0.in .0in 8.in 8.in] 
       {\plots/srcs01_000072/note/fig_45}
       \caption[]{
         Exp1 fits: values of fitted parameters. Distributions in P0, P1, and P2 
         for the two channels are overlayed
       }
       \label{fig:srcs01_000072_fig_45}
  \end{center}
\end{figure}


Distribuions  in $\chi^2/N_{DOF}$ and the number of fit points are presented 
in Fig. \ref{fig:srcs01_000072_fig_41}.

\begin{figure}[h!]
  \begin{center}
%    \vspace*{-2.8cm}
    \includegraphics [width=1.0\textwidth] % , clip=true, viewport=0.in .0in 8.in 8.in] 
       {\plots/srcs01_000072/note/fig_41}
       \caption[]{
         Exponential fits: distributions in $\chi^2/N_{DOF}$ and the number of fit points 
       }
       \label{fig:srcs01_000072_fig_41}
  \end{center}
\end{figure}


Distribution in fractional part of T0 presented in in Fig.\ref{fig:srcs01_000072_fig_44} 
doesn't show any statistically significant nonuniformities.

\begin{figure}[h!]
  \begin{center}
%    \vspace*{-2.8cm}
    \includegraphics [width=1.1\textwidth] % , clip=true, viewport= 0.in .0in 8.in 7.in] 
       {\plots/srcs01_000072/note/fig_44}
       \caption[]{Figure 5: }
       \label{fig:srcs01_000072_fig_44}
  \end{center}
\end{figure}

Timing resolution vs chi2 cut for exponential fit is shown in Fig. \ref{fig:srcs01_000072_fig_43}.
One can see that the waveform parameterization improves resolution by about 15\%.

\begin{figure}[h!]
  \begin{center}
%    \vspace*{-2.8cm}
    \includegraphics [width=1.0\textwidth] % , clip=true, viewport=0.in .0in 8.in 8.in] 
       {\plots/srcs01_000072/note/fig_43}
       \caption[]{CTR resolution: EXP1 vs CFD}
       \label{fig:srcs01_000072_fig_43}
  \end{center}
\end{figure}


Finally, Fig. \ref{fig:srcs01_000072_fig_53} compares timing resolutions resulting 
from two different fits - we find that they are close enough.

\begin{figure}[h!]
  \begin{center}
%    \vspace*{-2.8cm}
    \includegraphics [width=1.0\textwidth] %, clip=true, viewport=0.in .0in 8.in 8.in] 
       {\plots/srcs01_000072/note/fig_53}
       \caption[]{timing resolution: GAUS vs EXP1}
       \label{fig:srcs01_000072_fig_53}
  \end{center}
\end{figure}

Consistency of the results obtained using very different parameterizations indicates 
waveform digitization allows to improve the timing resolution, the improvement we observe 
is of the order of 15\%.

%%%%%%%%%%%%%%%%%%%%%%%%%%%%%%%%%%%%%%%%%%%%%%%%%%%%%%%%%%%%%%%%%%%%%%%%%%%%%%%
\newpage
\section{Fixed Shape Fit}

Another option for fitting is to use a fixed pulse shape, where the pulse shape could 
be derived directly from the data. For a given channel, an averaged pulse shape for 
non-scattered photons is determined and , after being normalized to a maximum of 1, 
is parameterized and used as a fit function. Advantage of this approach is that it uses 
a 1-parameter fit, time offset of the fit function, and completely avoids dealing with 
the analythical parameterization of the pulse shape.

The main disadvantage of this approach is that it doesn't take into account statistical 
fluctuations in the pulse development. If those are significant, they could affect the 
timing resolution. Also, for different channels the pulse shapes may be different and 
their time-dependence has to be studied.

However, we test its resolution.

Fig.\ref{fig:spln_shape_display_ev_0001} gives an example of the fit.

\begin{figure}[h!]
  \begin{center}
%    \vspace*{-2.8cm}
    \includegraphics [width=1.0\textwidth] %, clip=true, viewport=0.in .0in 8.in 8.in] 
       {\plots/srcs01_000072/note/spln_shape_display_ev_0001}
       \caption[]{Waveform fit with the parameterized average pulse shapes}
       \label{fig:spln_shape_display_ev_0001}
  \end{center}
\end{figure}

The early, 0.02-0.25, part of the waveform is fit with the exponential .

Figure \ref{fig:fig_71} shows distributions in the number of fit points and chi2 of the fit.

\begin{figure}[h!]
  \begin{center}
%    \vspace*{-2.8cm}
    \includegraphics [width=0.9\textwidth] %, clip=true, viewport=0.in .0in 8.in 8.in] 
       {\plots/srcs01_000072/note/fig_71}
       \caption[]{
         Average pulses and distributions in the number of fit points and $\chi^2/NDOF$ 
       }
       \label{fig:fig_71}
  \end{center}
\end{figure}

Figure \ref{fig:fig_72} shows distribution in  the coincidence time resolution for the 
average waveform fit. Its $\sigma = 138$ ps, about 40\% worse than the best result 
obtained with the gaussian parameterization of the leading edge. This difference 
indicates that statistical variations of the pulse shape are important and ignoring 
them significantly affects the timing reslution.

\begin{figure}[h!]
  \begin{center}
%    \vspace*{-2.8cm}
    \includegraphics [width=1.0\textwidth] %, clip=true, viewport=0.in .0in 8.in 8.in] 
       {\plots/srcs01_000072/note/fig_72}
       \caption[]{
         Distribution in $\Delta T_{01} = T_0 - T_1$, Coinsidence Timing Resolution CTR $\simeq$ 138 ps
       }
       \label{fig:fig_72}
  \end{center}
\end{figure}



{\bf what is coming after that ?}

%%%%%%%%%%%%%%%%%%%%%%%%%%%%%%%%%%%%%%%%%%%%%%%%%%%%%%%%%%%%%%%%%%%%%%%%%%%%%%%
\section {Conclusions }

\begin{itemize}
\item
  using the waveform digitization we measure the CTR below 240 ps FWHM
\item 
  compared to CFD-based determination of the pulse timing, fitting of the 
  waveform leading edge and determining the pulse timing based on the 
  fit results improves the timing resolution by about 15\%
\item 
  different parameterizations of the pulse leading edge result in close 
  timing resolutions. So far, the best timing resolution achieved 
  used gaussian parameterisation of the waveform leading edge described 
  in Section \ref{sec:gaussian_fit}. We also note that, out of three 
  different parameterizations tried, the gaussian fit gives the best 
  average value of $\chi^2/NDOF$.
\item 
  given limited photostatistics, event-to-event fluctuations 
  of the pulse leading edge shape are important, ignoring them and 
  using the ``average pulse shape'' for the pulse timing determination 
  degrades the timing resolution by as much as $\sim 40\%$.
\end{itemize}


%----------------------------------
%
\begin{thebibliography}{99}

% Derenzo PET scanner space resolution formula 
\bibitem{derenzo1997}
   W.Moses et at, IEEE Transactions on Nuclear Science NS-44, pp. 1487–1491, 1997 

\bibitem{van_dam_2011} 
  H.T. van Dam et al, IEEE Transactions on Nuclear Science, v.58, 5, p 2139, 2011

\bibitem{schaart_2012} 
  D.Schaart et al, talk at PHODET 2012, \\
  https://indico.cern.ch/getFile.py/access?contribId=46\&sessionId=3\&resId=0\&materialId=slides\&confId=164917

\bibitem{sipm_angular_efficiency}
   http://web.physik.rwth-aachen.de/\~ ~hebbeker/theses/schumacher\_bachelor.pdf
\end{thebibliography}



\section {Apppendix 1: Pedestal Subtraction }
\label{sec:appendix_1}

%
First, the voltage offset is calculated as a mean over channels 50-150. 
The offset is subtracted from the waveform. A channel \tzeroa\, corresponding 
to the 20 mV pulse height, is found by backward scanning of the leading edge 
and the approximate value of pulse time \tzeroa\ is determined in the linear 
approximation as

$$
   T_0^a ~=~ T_{20} ~-~ {{V(T_{20})}\over dV/dT};  
   ~~ {\rm where} ~~ dV/dT ~=~ {{(V(T_{20}+2) ~-~ V(T_{20}-2))} \over 4}
$$

After \tzeroa\ is determined, the pedestal is recalculated over the 10ns range 
directly preceding the pulse, using channels from \tzeroa-60 to \tzeroa-10, and 
subtracted  from the waveform.

\begin{figure}[h!]
  \begin{center}
%    \vspace*{-2.8cm}
    \includegraphics [width=.9\textwidth, clip=true, viewport=0.in .0in 8.in 6.in] 
       {\plots/srcs01_000072/run_072_fig_003_ped_c2p}
       \caption[]{
         Fitted pedestals, in the top row, and standard deviations of the pedestal fit, 
         in the bottom row. Compared to the typical noise of 0.4-0.5 mV, the pedestals 
         are stable over the course of the run. \\
         {\bf 1. look more closely at events where pedestals are 1 mV away from the mean} \\
         {\bf 2. look at events with RMS around 2.5-3} \\
         {\bf 3. c2p's in the histogram titles are, in fact, not $\chi^2$'s but $\sigma$'s}
       }
       \label{fig:run_072_fig_003_ped_c2p}
  \end{center}
\end{figure}

As shown in Fig.~\ref{fig:run_072_fig_003_ped_c2p}, the mean pedestals are stable 
within the run compared to the noise. Average noise,  determined as the waveform RMS 
calculated over the pedestal area, is about 0.4-0.5 mV.

\end{document}

%%%%%%%%%%%%%%%%%%%%%%%%%%%%%%%%%%%%%%%%%%%%%%%%%%%%%%%%%%%%%%%%%%%%%%%%%%%%%%%
% end of the document
%%%%%%%%%%%%%%%%%%%%%%%%%%%%%%%%%%%%%%%%%%%%%%%%%%%%%%%%%%%%%%%%%%%%%%%%%%%%%%%
