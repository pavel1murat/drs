%%%%%%%%%%%%%%%%%%%%%%%%%%%%%%%%%%%%%%%%%%%%%%%%%%%%%%%%%%%%%%%%%%%%%%%%%%%%%%%
% files to use
% 
\documentclass[12pt]{article}
\usepackage{color}
\usepackage{graphicx}
\usepackage{epsf}
\usepackage{setspace}  % for doublespacing
%\linespread{2.}

\usepackage{bm}        % for math
\usepackage{ulem}      % for strike out
\usepackage{color}     %
\usepackage{amssymb}   % for math
%\usepackage{multicol}   % needed for some tables
% \usepackage{multirow}
\usepackage{epstopdf}  %added for MAC compiler
%%%%%%%%%%%%%%%%%%%%%%%%%%%%%%%%%%%%%%%%%%%%%%%%%%%%%%%%%%%%%%%%%%%%%%%%%%%%%%%
\begin{document}
%%%%%%%%%%%%%%%%%%% \input{psfig}
% \input{commands} 
%%%%%%%%%%%%%%%%%%%%%%%%%%%%%%%%%%%%%%%%%%%%%%%% local new commands

\newcommand {\chisq}         {\mbox{$\chi^2$}}
\newcommand {\rsw}           {\mbox{$R_{sw}$}}
\newcommand {\rwp}           {\mbox{$R_{wp}$}}
\newcommand {\DeltaPhi}      {\mbox{$\Delta\phi$}}
\newcommand {\ecal}          {\mbox{$E_{cal}$}}
\newcommand {\et}            {\mbox{${E_T}$}}
\newcommand {\etcal}         {\mbox{${E_T}_{cal}$}}
\newcommand {\etvis}         {\mbox{${E_T}_{vis}$}}
\newcommand {\evis}          {\mbox{$E_{vis}$}}
\newcommand {\ewire}         {\mbox{$E_{wire}$}}
\newcommand {\etauid}        {\mbox{$\epsilon_{ID}^{\tau}$}}
\newcommand {\estrip}        {\mbox{$E_{strip}$}}
\newcommand {\etcorr}        {\mbox{$E_T^{corr}$}}
\newcommand {\gt}            {\mbox{$>$}}
\newcommand {\gea}           {\mbox{$>=$}}
\newcommand {\invfb}         {\mbox{$fb^{-1}$}}
\newcommand {\invpb}         {\mbox{$pb^{-1}$}}
\newcommand {\LaBrThree}     {\mbox{$\rm LaBr_3$}}
\newcommand {\lea}           {\mbox{$<=$}}
\newcommand {\lt}            {\mbox{$<$}}
\newcommand {\met}           {\mbox{${\not \! E}_{T}$}}

\newcommand {\mmsq}          {\mbox{${\rm mm^2}$}}
\newcommand {\mmcube}        {\mbox{${\rm mm^3}$}}

\newcommand {\NtrkTen}       {\mbox{$N_{trk}^{10}$}}
\newcommand {\NtrkTenThirty} {\mbox{$N_{trk}^{10-30}$}}
\newcommand {\ppbar}         {\mbox{$p\bar{p}$}}
\newcommand {\ttbar}         {\mbox{$t\bar{t}$}}
\newcommand {\pt}            {\mbox{$p_{T}$}}
\newcommand {\ptrack}        {\mbox{$P_{track}$}}
\newcommand {\vismass}       {\mbox{$M_{track+\pi^0}$}}
\newcommand {\yrec}          {\mbox{$Y_{rec}$}}
\newcommand {\zrec}          {\mbox{$Z_{rec}$}}

\newcommand{\Pt}{\ensuremath {p_{\rm{t}}}}
\newcommand{\Et}{\ensuremath{E_{T}}}

%%% \newcommand {\upar}               {\mbox{$U_{||}$}}
%%%% $ \sum \vec{p_t} $

\newcommand {\plots} {/home/murat/figures/mcrystal/note}

%%%%%%%%%%%%%%%%%%%%%%%%%%%%%%%%%%%%%%%%%%%%%%%%%%%%%%%%%%%%%%%%%%%%%%%%%%%%%%%
\begin{titlepage}

  \begin{flushright}
    \bf Fermilab Technical note  \\
    version 1.01
    \today
  \end{flushright}
  
  \vspace{1cm}
  
  \begin{center} {
      \Large \bf
      Coordinate and Depth-Of-Interaction Resolution of Large Scintillating Crystals 
      with 2-sided SiPM-based Readout: a Monte Carlo Study
    }

    \vspace{1cm}

    { 
      P.Murat(Fermilab)
    }

 \end{center}

  \vspace{1cm}

  \begin{abstract}

    We present results of a Monte Carlo (MC) study of coordinate and depth-of-interaction 
    resolution of relatively large, several centimeters in size, scintillating crystals 
    with 2-sided SiPM-based readout. 

    When a single non-pixelated crystal is read out from two opposite sides by SiPM matrices, 
    the coordinate resolution achieved can be significantly better than the crystal or SiPM size.

    In particular, for a benchmark point with 20x20x10\mmcube\ LYSO crystal read out by 
    2 matrices of 4x4\mmsq\ SiPMs, one could achieve accuracy in the coordinate 
    measurement of the order of 0.5mm FWHM. The corresponding accuracy in measuring 
    the depth of interaction could be better than 1mm. 

    The resolution effectively depends on the crystal geometry, the ratio of the crystal 
    thickness to its transverse size, H/D, significantly improving at small H/D.

    We discuss implication of these results for design of the high-resolution PET 
    scanners.

  \end{abstract}

\end{titlepage} 

{\tableofcontents}

%%%%%%%%%%%%%%%%%%%%%%%%%%%%%%%%%%%%%%%%%%%%%%% Introduction
\newpage
\section{Introduction}

Most modern high-resolution PET scanner designs are based on pixelated scintillation 
arrays with each pixel read out by a single photodetector. In this case the coordinate 
resolution can be approximated by the following formula
$$
     \Gamma ~=~ 1.25 \sqrt{ ~ {d \over 2}^2 + (0.0022D)^2 + R^2 + b^2} ,
$$
, where d is the pixel size, D - the PET detector radius, R - the positron 
range, and b - an additional constant accounting for residual misalignment 
etc, often referred to as Derenzo's resolution formula \cite{derenzo1997}.

It has been shown that in case of non-pixelated, or monolithic, crystals read 
out by several position-sensitive photodetectors, for example a silicon 
photomultiplier (SiPM) matrix, it is possible to achieve resolutions 
significantly better than the crystal or SiPM size \cite{van_dam_2011}. 
However, the data processing technique described in\cite{van_dam_2011} 
is almost prohibitevely CPU-intensive.

Compactness of SiPM's allows to implement a double-sided readout with a single 
scintillating crystal read out by two SiPM matrices from the opposite sides, 
as shown in Fig.~\ref{fig:pet32-drawings-2-sided-readout}. 
%
At a cost of doubling the number of channels, the double-sided readout allows 
to use simple and fast reconstruction techniques to achieve submillimeter 
coordinate resolutions, outperforming the pixelated arrays on a cost/resolution 
basis.
%
In addition, double-sided readout allows the depth of interaction measurements. 

Coupling multiple SiPM's to one crystal reduces the amount of light detected 
by each SiPM and thus affects the accuracy of the timing measurements. 
It has been shown, however, that using digital SiPMs allows to achieve 
timing resolution CRT \lt\ 200ps FWHM with 20mm thick LSO crystals \cite{schaart_2012}. 

In this note, we perform a Monte Carlo study of the coordinate resolution of 
relatively large, several centimeters in size, crystals, and describe a simple 
and fast reconstruction techniques including reconstruction of the depth 
of interaction (DOI).

%%%%%%%%%%%%%%%%%%%%%%%%%%%%%%%%%%%%%%%%%%%%%%%%%%%%%%%%%%%%%%%%%%%%%%%%%%%%%%%
\newpage
\section{MC Simulation}

The simulated detector setup consists of a box-shaped scintillating crystal 
read out from the two opposite sides by the SiPM matrices. 
The crystal cross section is varied in between 20x20\mmsq\ and 40x40\mmsq. 
The crystal thickness is varied from 10m to 30mm to cover the range of approximately 
one to three radiation lengths for the heavy neorganic scintillators of interest. 

We assume that a 511 KeV photon from the $e^+e^-$ annihilation hits the 
scintillating crystal and that, although this is a simplification, the 
photon interaction point is distributed uniformly within the crystal 
volume. We also consider only the photoabsorption of the annihilation 
photons and ignore their Compton scattering inside the crystal and the size 
of the EM shower developed in the crystal, assuming all optical photons to be  
produced in the same space point.

As the spatial resolution depends on the light yield, we consider two different 
scintillators,  lutetium yttrium orthosilicate (LYSO)) and \LaBrThree , with 
the light yields of 26000 and 60000 optical photons per MeV correspondingly.

The coordinate system is chosen such that the SiPM matrices are located on the 
two Z-sides of the crystal, as shown in Fig.~\ref{fig:pet32-drawings-2-sided-readout}.

\begin{figure}[h!]
  \begin{center}
%    \vspace*{-2.8cm}
    \includegraphics [width=.6\textwidth, clip=true, viewport=0.in .0in 8.in 8.in] 
       {\plots/pet32-drawings-2-sided-readout}
       \caption[]{A typical event, 20x20mm$^2$ crystal, 4x4mm$^2$ SiPM matrix}
       \label{fig:pet32-drawings-2-sided-readout}
  \end{center}
\end{figure}

The reflection efficiency from the crystal walls is taken to be 95\%, 
independent on the photon wavelength. 
%
The SiPM photon detection efficiency used by the simulation is 25\%, 
also independent on the optical photon wavelength, and the angular 
dependence of the SiPM photon detection efficiency is taken 
from \cite{sipm_angular_efficiency}.

The simulated SiPM matrices consist of 2x2\mmsq\ or 4x4\mmsq\ SiPMs. 
Spacing between the SiPMs is assumed to be zero, and theSiPM  matrix 
fully covers the corresponding crystal side.

%%%%%%%%%%%%%%%%%%%%%%%%%%%%%%%%%%%%%%%%%%%%%%%%%%%%%%%%%%%%%%%%%%%%%%%%%%%%%%
\newpage
\section{Reconstruction}

When a primary 511 KeV photon interacts inside the scintillator, its energy 
gets re-emitted in a form of multiple optical photons detected by the SiPM's.
The SiPM pulse height distribution is sensitive to the location of the 
interaction point. 

Figure ~\ref{fig_6} shows an ``event display'' of a typical 
event for 20x20x20\mmcube\ LYSO crystal read our by 5x5 matrices of 4x4\mmsq\ 
SiPMs. Numbers of entries per cell/bin on the projections correspond to the 
number of detected optical photons. For the displayed event the Z coordinate 
of the interaction point is close to the Z = -1cm side, and it is clear 
from Figure ~\ref{fig_6} that the center of gravity of the pulse height 
distribution on that side gives a good estimate of the X and Y coordinates 
of the interaction point. It is also clear that the charge distribution on 
the opposite side is much less sensitive ot the coordinates of the interaction 
point. 

\begin{figure}[h!]
  \begin{center}
%    \vspace*{-2.8cm}
    \includegraphics [width=.9\textwidth, clip=true, viewport=0.in .0in 8.in 8.in] 
       {\plots/fig_6}
       \caption[]{A typical event, 20x20mm2 crystal, 2x2mm2 SiPM}
       \label{fig_6}
  \end{center}
\end{figure}

We use the 1D projections of the distributions in Fig.~\ref{fig_6} to reconstruct 
the mean values of the charge distributions:

$$
  <X> = {\Sigma { X_i \cdot Q_i} / \Sigma{Q_i}}, ~~~~~~~ <Y> = {\Sigma { Y_i \cdot Q_i} / \Sigma{Q_i}}, 
$$

and their respective widths (the second momenta of the charge distributions) 

$$
  {\sigma_x}^2 ~=~ <X^2>-<X>^2 , ~~~~~~~~~ {\sigma_y}^2 ~=~ <Y^2>-<Y>^2
$$

Fig.~\ref{fig_8} shows correlation between the width of the charge distribution 
$\sigma_R = \sqrt{\sigma_x^2 + \sigma_y^2}$ reconstructed the Z=ZMAX side and the 
Z coordinate of the interaction point. 

\begin{figure}[h!]
  \begin{center}
%    \vspace*{-2.8cm}
    \includegraphics [width=.6\textwidth, clip=true, viewport=1.in .0in 8.0in 8.0in] 
       {\plots/fig_8}
       \caption[]{
         Width of the charge distribution averaged over $X_V$ and $Y_V$, 
         and plotted vs the $Z_V$. 20x20x20 mm3 LYSO crystal, 4x4mm SiPM.
       }
       \label{fig_8}
  \end{center}
\end{figure}

The larger values of Z correspond to smaller cluster widths reconstructed on Z=ZMAX side 
and larger cluster widths reconstructed on the Z=ZMIN side. For events with ZMAX-Z < 2mm 
very often all the light gets collected by only one SiPM, so the calculated width of the 
charge distribution is zero.

To reconstruct X and Y coordinates we on the 2 z sides are weighted with the weights defined 
by the corresponding charge distribution widths:

$$
x = x_1 \cdot w_1 + x_2 \cdot w_2 ~~ ;~~ w_i = {{1 \over \sigma_i^2} \cdot {{\sigma_1^2 \sigma_2^2} \over {\sigma_1^2 + \sigma_2^2}}}
$$

where $X_i$ is a coordinate reconstructed on the corresponding Z side.

The Z coordinate is reconstructed in a similar way. Assuming $\Delta Z$ is the crystal 
half-thickness, the weighting gives: 

$$
   z = \Delta Z \cdot (\sigma_1^2-\sigma_2^2) / (\sigma_1^2 + \sigma_2^2)
$$

%%%%%%%%%%%%%%%%%%%%%%%%%%%%%%%%%%%%%%%%%%%%%%%%%%%%%%%%%%%%%%%%%%%%%%%%%%%%%%%
% \newpage
\section {Resolution in Depth Of Interaction}

Weight-based technique of reconstructing the Z-coordinate introduces a systematic 
offset between the reconstructed coordinate and the true one, called differential 
non-linearity.

The differential non-linearity, shown in Fig.~\ref{fig_9}, left, can be accounted 
for by either the experimental or MC-based calibration. The corrected distribution 
shown in Fig.~\ref{fig_9} (right) has non-gaussian tails, however the resolution 
in the reconstructed depth of interaction, DOI, is about 1.8 mm FWHM. 
Although this number is just a MC estimate, it is much better than the currently 
achieved resolution in DOI, which is typically of the order of the thickness of 
the crystal used.

\begin{figure}[h!]
  \begin{center}
%    \vspace*{-2.8cm}
    \includegraphics [width=.9\textwidth, clip=true, viewport=0.in .0in 8.in 4.5in] 
       {\plots/fig_9}
       \caption[]{
         a) Difference between the reconstructed and true Z coordinates plotted vs 
         the true Z coordinate. 20x20x20 LYSO crystal, 4x4mm SiPM. Non-linearity 
         effect is seen.
         b) distribution in Z-residuals, corrected for non-linearity. The distribution 
         has non-gaussian tails, the resolution in the reconstructed depth of interaction 
         is about 1.8mm FWHM.
       }
       \label{fig_9}
  \end{center}
\end{figure}


%%%%%%%%%%%%%%%%%%%%%%%%%%%%%%%%%%%%%%%%%%%%%%%%%%%%%%%%%%%%%%%%%%%%%%%%%%%%%%
\newpage
\section {Resolution in X and Y}

The simulated crystal has a square cross section, and the resolutions in X and Y 
coordinates are supposed to be the same. We, therefore, discuss only resolution in X 
Figure \ref{fig_21} shows distributions in X residuals, $\Delta X = X_{reco}-X_{true}$, 
plotted vs the reconstructed X coordinate for 20x20x20 \mmcube\ crystal and 4x4 \mmsq\ 
SiPM-based readout. 

\begin{figure}[h!]
  \begin{center}
%    \vspace*{-2.8cm}
    \includegraphics [width=.9\textwidth, clip=true, viewport=0.in .0in 8.in 6.in] 
       {\plots/fig_21}
       \caption[]{
         Residuals $\Delta X$ vs $X_{rec}$ , averaged over the crystal 
         volume for 20x20x20 LYSO crystal and 4x4mm SiPM's.
       }
       \label{fig_21}
  \end{center}
\end{figure}

The distribution has several characteristic features. 

\begin{itemize}
\item
  given the described algorithm, one can't reconstruct the value of X-coordinate greater 
  than 0.8 cm
\item 
  for $X > 0.6$ the profile distribution peaks in the edge bin, 
  so in this region the reconstructed coordinate doesn't depend 
  on the true one.
\item 
  the best resolution is achieved around $|X| = 0.4$, the region where 
  the cluster is centers in the bin, minimizing the calculated $\sigma$.
\item 
  the line around 0 corresponds to events with large |Z|, for which 
  one of the reconstructed values of $\sigma$ is equal to zero.
\end{itemize}

Fig.~\ref{fig_71} which demonstrates $\Delta Y$ dependence on $Y_{rec}$. shows that while 
the reconstructed $Y_{rec}$, on average, doesn't depend on Z, 
however and the reconstructed $Z_{rec}$ doesn't n average depend 
on Y, we observe typical dependencies of $Y_{rec}$ on $Y_{true}$ and $Z_{rec}$ on $Z_{true}$ 
and introduce average corrections.

\begin{figure}[h!]
  \begin{center}
%    \vspace*{-2.8cm}
    \includegraphics [width=.9\textwidth, clip=true, viewport=0.in .0in 8.in 8.in] 
       {\plots/fig_71}
       \caption[]{Resolution for X=0 Y=0, error bars represent widths of the 
       corresponding residual distributions}
       \label{fig_71}
  \end{center}
\end{figure}


Averaging the corrected residuals over the crystal volume results in the X- and Y- resolutions of 
about 1.8mm FWHM, as shown in Figure \ref{fig_5}, left. This result is consistent with the results 
of 

Also shown in Figure \ref{fig_5} is the resolution in the Z coordinate, the depth of 
interaction. Even with some remaining systematic effects not taken into account, the resolution 
in the depth of interaction, DOI, is about 3mm FWHM, about 7 times better than the crystal 
size of 20mm

\begin{figure}[!ht]
  \begin{center}
%    \vspace*{-2.8cm}
    \includegraphics [width=.9\textwidth, clip=true, viewport=0.in .0in 8.in 8.in] 
       {\plots/fig_5}
       \caption[]{Corrected residuals vs the reconstructed coordinates - the systematics are still seen}
       \label{fig_5}
  \end{center}
\end{figure}


%%%%%%%%%%%%%%%%%%%%%%%%%%%%%%%%%%%%%%%%%%%%%%%%%%%%%%%%%%%%%%%%%%%%%%%%%%%%%%%
\section{Results}

Resolutions are presented in Figure \ref{fig_11}.

\begin{figure}[ht!]
  \begin{center}
%    \vspace*{-2.8cm}
    \includegraphics [width=.9\textwidth, clip=true, viewport=0.in .0in 8.in 8.in] 
       {\plots/fig_11}
       \caption[]{Resolutions}
       \label{fig_11}
  \end{center}
\end{figure}


Resolution in Y2 for different geometries


\begin{figure}[ht!]
  \begin{center}
%    \vspace*{-2.8cm}
    \includegraphics [width=.9\textwidth, clip=true, viewport=0.in .0in 8.in 8.in] 
       {\plots/fig_12}
       \caption[]{Resolutions}
       \label{fig_12}
  \end{center}
\end{figure}


Resolution in Z (depth of interaction) for different geometries

\begin{figure}[ht!]
  \begin{center}
%    \vspace*{-2.8cm}
    \includegraphics [width=.9\textwidth, clip=true, viewport=0.in .0in 8.in 8.in] 
       {\plots/fig_13}
       \caption[]{Resolutions}
       \label{fig_13}
  \end{center}
\end{figure}


Resolution in Y4 for different geometries

\begin{figure}[ht!]
  \begin{center}
%    \vspace*{-2.8cm}
    \includegraphics [width=.9\textwidth, clip=true, viewport=0.in .0in 8.in 8.in] 
       {\plots/fig_14}
       \caption[]{Resolutions}
       \label{fig_14}
  \end{center}
\end{figure}




%%%%%%%%%%%%%%%%%%%%%%%%%%%%%%%%%%%%%%%%%%%%%%%%%%%%%%%%%%%%%%%%%%%%%%%%%%%%%%%
\section {Conclusions}



%----------------------------------
%
\begin{thebibliography}{99}

% Derenzo PET scanner space resolution formula 
\bibitem{derenzo1997}
   W.Moses et at, IEEE Transactions on Nuclear Science NS-44, pp. 1487–1491, 1997 

\bibitem{van_dam_2011} 
  H.T. van Dam et al, IEEE Transactions on Nuclear Science, v.58, 5, p 2139, 2011

\bibitem{schaart_2012} 
  D.Schaart et al, talk at PHODET 2012, \\
  https://indico.cern.ch/getFile.py/access?contribId=46\&sessionId=3\&resId=0\&materialId=slides\&confId=164917

\bibitem{sipm_angular_efficiency}
   http://web.physik.rwth-aachen.de/\~ ~hebbeker/theses/schumacher\_bachelor.pdf
\end{thebibliography}


\end{document}

%%%%%%%%%%%%%%%%%%%%%%%%%%%%%%%%%%%%%%%%%%%%%%%%%%%%%%%%%%%%%%%%%%%%%%%%%%%%%%%
% end of the document
%%%%%%%%%%%%%%%%%%%%%%%%%%%%%%%%%%%%%%%%%%%%%%%%%%%%%%%%%%%%%%%%%%%%%%%%%%%%%%%
