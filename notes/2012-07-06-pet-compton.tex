% $Header: /home/camac/cvs/drs/notes/2012-07-06-pet-compton.tex,v 1.1 2012/07/06 16:29:35 camac Exp $

\documentclass{beamer}[10pt]

% This file is a solution template for:

% - Talk at a conference/colloquium.
% - Talk length is about 20min.
% - Style is ornate.



% Copyright 2004 by Till Tantau <tantau@users.sourceforge.net>.
%
% In principle, this file can be redistributed and/or modified under
% the terms of the GNU Public License, version 2.
%
% However, this file is supposed to be a template to be modified
% for your own needs. For this reason, if you use this file as a
% template and not specifically distribute it as part of a another
% package/program, I grant the extra permission to freely copy and
% modify this file as you see fit and even to delete this copyright
% notice. 


\mode<presentation> {
  % \usetheme{Warsaw}
  \usetheme{Boadilla}
%  \usetheme{Darmstadt}
%  \usetheme{Berkeley}
  % or ...

%  \setbeamercovered{transparent}
  % or whatever (possibly just delete it)
}


\usepackage[english]{babel}
% or whatever

\usepackage[latin1]{inputenc}
% or whatever

\usepackage{times}
\usepackage{graphics}
\usepackage{subfigure}

%  \usepackage{pgfpages}
% \pgfpagesuselayout{2 on 1}[letterpaper,border shrink=2mm,landscape]

% make beamer and subfigure work well
\makeatletter
\def\ext\@subfigure{lof}
\makeatother

% \usepackage{geometry}
% \geometry{letterpaper,landscape}

% \usepackage{caption}
\usepackage[T1]{fontenc}


% Or whatever. Note that the encoding and the font should match. If T1
% does not look nice, try deleting the line with the fontenc.

\title{
  \underline \small {\bf Events with Compton scattering and the strip line -based readout }
}

%% \subtitle{ 
%%   \vspace{0.5in}
%% %  {\underline {$ZZ \to 4$ Leptons Channel}} , \\
%%   {\small What one can do}
%% }


%% \institute{\inst{1}Fermilab }
% \institute{\inst{1}Univ of Athens, \inst{2}Fermilab , \inst{3}Univ of Glasgow}

\author{
  \fontseries{s}
  \fontsize{6}{8}
  \selectfont
%  {\bf P.Murat  \inst{2} }
  S.One, S.Two, {\bf P.Murat}
}

\date{\today}
%\date{Aug 18 2011}

% - Use the \inst command only if there are several affiliations.
% - Keep it simple, no one is interested in your street address.

% - Either use conference name or its abbreviation.
% - Not really informative to the audience, more for people (including
%   yourself) who are reading the slides online

\subject{Experimental report}
% This is only inserted into the PDF information catalog. Can be left
% out. 

% If you have a file called "university-logo-filename.xxx", where xxx
% is a graphic format that can be processed by latex or pdflatex,
% resp., then you can add a logo as follows:

%%% \pgfdeclareimage[height=0.5cm]{university-logo}{checks\_reorder.pdf}
%%% \logo{\pgfuseimage{university-logo}}

% Delete this, if you do not want the table of contents to pop up at
% the beginning of each subsection:
%%\AtBeginSubsection[] {
%%%------------------------------------------------------------------------------
%% \begin{frame}<beamer>{\underline{Outline}}
%%    % \tableofcontents[currentsection,currentsubsection]
%%  \tableofcontents[currentsection]
%% \end{frame}
%%}
%%

% If you wish to uncover everything in a step-wise fashion, uncomment
% the following command: 

% \beamerdefaultoverlayspecification{<+->}



%%%%%%%%%%%%%%%%%%%%%%%%%%%%%%%%%%%%%%%%%%%%%%%%%%%%%%%%%%%%%%%%%%%%%%%%%%%%%%%
% COMMANDS
%%%%%%%%%%%%%%%%%%%%%%%%%%%%%%%%%%%%%%%%%%%%%%%%%%%%%%%%%%%%%%%%%%%%%%%%%%%%%%%
\newcommand {\eemm}          {\mbox{$ee\mu\mu$}}
\newcommand {\et}            {\mbox{$E_T$}}
\newcommand {\etcorr}        {\mbox{$E_T^{corr}$}}
\newcommand {\gevcsq}        {\mbox{$GeV\!/c^2$}}
\newcommand {\gt}            {\mbox{$>$}}
\newcommand {\invfb}         {\mbox{$fb^{-1}$}}
\newcommand {\invfbrm}       {\mbox{$\rm fb^{-1}$}}
\newcommand {\invpb}         {\mbox{$pb^{-1}$}}
\newcommand {\invpbrm}       {\mbox{$\rm pb^{-1}$}}
\newcommand {\jpsi}          {\mbox{$J/\psi$}}
\newcommand {\lt}            {\mbox{$<$}}
\newcommand {\met}           {\mbox{${\not\! E}_{T}$}}
\newcommand {\mmmm}          {\mbox{$\mu\mu\mu\mu$}}
\newcommand {\mzz}           {\mbox{$M_{ZZ}$}}
\newcommand {\pb}            {\mbox{\rm\,pb}}
\newcommand {\ppbar}         {\mbox{$p\bar{p}$}}
\newcommand {\pt}            {\mbox{$p_T$}}
\newcommand {\red}           {\color{red}}
\newcommand {\stat}          {\mbox{\rm (stat.)}}
\newcommand {\statsys}       {\mbox{\rm (stat.+syst.)}}
\newcommand {\syst}          {\mbox{\rm (syst.)}}
\newcommand {\wpigamma}      {\mbox{$W^{\pm} \rightarrow \pi^{\pm} \gamma$}}
\newcommand {\wenu}          {\mbox{$W^{\pm} \rightarrow e^{\pm} \nu$}}
\newcommand {\wlnu}          {\mbox{$W^{\pm} \rightarrow l^{\pm} \nu$}}
\newcommand {\wmunu}         {\mbox{$W^{\pm} \rightarrow \mu^{\pm} \nu$}}
\newcommand {\wtaunu}        {\mbox{$W^{\pm} \rightarrow \tau^{\pm} \nu$}}
\newcommand {\zpsigamma}     {\mbox{$    Z^{0} \rightarrow J/\psi \gamma$}}
\newcommand {\zpigamma}      {\mbox{$\rm Z^{0} \rightarrow \pi^{0} \gamma$}}
\newcommand {\zee}           {\mbox{$\rm Z^0 \rightarrow e^+ e^-$}}
\newcommand {\zmumu}         {\mbox{$\rm Z^0 \rightarrow \mu^+ \mu^-$}}
\newcommand {\ztautau}       {\mbox{$\rm Z^0 \rightarrow \tau^+ \tau^-$}}
\newcommand {\zll }          {\mbox{$Z       \rightarrow l^{+}l^{-}$}}
\newcommand {\zupsgamma}     {\mbox{$    Z^0 \rightarrow \Upsilon \gamma$}}
\newcommand {\zzx }          {\mbox{$X       \to ZZ$}}
\newcommand {\zzllll}        {\mbox{$ZZ \to \ell^+ \ell^- \ell^+ \ell^-$}}
\newcommand {\zzllnn}        {\mbox{$ZZ \to \ell^+ \ell^- \nu \nu$}}
\newcommand {\zzlljj}        {\mbox{$ZZ \to \ell^+ \ell^- j j$}}

\newcommand {\plots}  {/home/camac/figures/}   % PET01
% \newcommand {\plots}  {/home/murat/figures/drs4}   % MURAT03


\begin{document}

%------------------------------------------------------------------------------
\begin{frame}
  \titlepage
\end{frame}

% \begin{frame}{\underline{Outline}}
%   \tableofcontents
%   % You might wish to add the option [pausesections]
% \end{frame}
% 
%------------------------------------------------------------------------------
% \AtBeginSection[] {
% % ------------------------------------------------------------------------------
%   \begin{frame}<beamer>{}
% %     \tableofcontents[currentsection,currentsubsection]
%     \tableofcontents[currentsection]
%   \end{frame}
% }
% 
% Structuring a talk is a difficult task and the following structure
% may not be suitable. Here are some rules that apply for this
% solution: 

% - Exactly two or three sections (other than the summary).
% - At *most* three subsections per section.
% - Talk about 30s to 2min per frame. So there should be between about
%   15 and 30 frames, all told.

% - A conference audience is likely to know very little of what you
%   are going to talk about. So *simplify*!
% - In a 20min talk, getting the main ideas across is hard
%   enough. Leave out details, even if it means being less precise than
%   you think necessary.
% - If you omit details that are vital to the proof/implementation,
%   just say so once. Everybody will be happy with that.

%%%%%%%%%%%%%%%%%%%%%%%%%%%%%%%%%%%%%%%%%%%%%%%%%%%%%%%%%%%%%%%%%%%%%%%%%%%%%%%
\section{Introduction}
%\subsection{What do we know about decays of the W's and Z's}

%------------------------------------------------------------------------------
\begin{frame}{ \underline{\small MPPC 11064 gain uniformity} }

  \fontseries{s}
  \fontsize{8}{10}
  \selectfont

  \begin{columns}
    \column{2.0in} {
      \begin{figure}[htbf]
  %  \vspace{-0.2in}
        %  \hspace{-0.4in}
        \includegraphics[width=0.9\textwidth]
        %, height=0.4\textheight, clip=true, viewport=0.0in 0.in 8in 7.7in] 
                        {\plots/Ortec_run95_q_000}
                        \caption{
  \fontseries{s}
  \fontsize{8}{10}
  \selectfont
                          Gain uniformity
                        }
      \end{figure}
    }
    \column{3.0in}{
      \begin{figure}[htbf]
        %  \vspace{-0.2in}
        %  \hspace{-0.4in}
        \includegraphics[width=0.9\textwidth]
        %, height=0.4\textheight, clip=true, viewport=0.0in 0.in 8in 7.7in] 
                        {\plots/Ortec_run95_q_000}
                        \caption{
  \fontseries{s}
  \fontsize{8}{10}
  \selectfont
                          after gain calibration: charge vs channel number
                        }
      \end{figure}
    }
  \end{columns}

  \begin{itemize}
  \item 
    gain uniformity at the level of 5\% (Phillips 776DC amplifiers: 6\%)
  \item 
    energy resolution: 12\%
  \end{itemize}

\end{frame}

%------------------------------------------------------------------------------
\begin{frame}{ \underline{\small Recover events with Compton scattering} }

  \fontseries{s}
  \fontsize{8}{10}
  \selectfont

\begin{figure}[htbf]
  %  \vspace{-0.2in}
  %  \hspace{-0.4in}
  \includegraphics[width=0.9\textwidth, height=0.7\textheight]
  %, clip=true, viewport=0.0in 0.in 4in 2.8in] 
                  {\plots/Ortec_run95_q_000}
\end{figure}

  \begin{itemize}
    \vspace{-0.2in}
  \item 
    Compton scattering event - a simple event display
  \item
    add up energies in the two channels
  \end{itemize}

\end{frame}


%------------------------------------------------------------------------------
\begin{frame}{ \underline{\small Recover events with Compton scattering - Contd} }

  \fontseries{s}
  \fontsize{8}{10}
  \selectfont

 \begin{figure}[htbf]
  %  \vspace{-0.2in}
  %  \hspace{-0.4in}
  \includegraphics[width=0.8\textwidth, height=0.7\textheight]
  %, clip=true, viewport=0.0in 0.in 4in 2.8in] 
                  {\plots/Ortec_run95_q_000}
                  \caption{Compton scattering}
\end{figure}

  \begin{itemize}
    \vspace{-0.2in}
  \item 
    total statistics in the photopeak increases by about 40\%
  \item 
    timing resolution to be investigated
  \end{itemize}

\end{frame}


%------------------------------------------------------------------------------
\begin{frame}{ \underline{\small Summary} }

  \fontseries{s}
  \fontsize{8}{10}
  \selectfont

  \begin{itemize}
  \item 
    excellent uniformity of Hamamatsu 11064 MPPCs
  \item 
    studies of the Compton scatterer recovery in progress
  \end{itemize}

\end{frame}

\end{document}
